\documentclass[12pt]{beamer}

\usepackage[utf8]{inputenc}
\usepackage{textgreek}
\usepackage{bm}

\usetheme{PaloAlto}
\usecolortheme{rose}

%Information to be included in the title page:
\title{Brachistochrone Problem}
\author{Rajeev Atla}
\institute{Physics Club}

\section{Administrivia}

\begin{document}

\frame{\titlepage}

\begin{frame}
\frametitle{This Section}
\begin{itemize}
    \pause
    \item More advanced
    \pause
    \item Goals
    \begin{itemize}
        \pause
        \item Get everyone to pass F=ma exam
        \pause
        \item USAPhO Qualifiers!!
    \end{itemize}
    \pause
    \item Prerequisites (recommended)
    \begin{itemize}
        \pause
        \item Taken/currently taking a physics class
        \pause
        \item Or...
        \pause
        \item Willingness to learn
    \end{itemize}
\end{itemize}

\end{frame}

\begin{frame}
\frametitle{PSA: Problems}
\begin{itemize}
    \item On classroom
    \pause
    \item Due date: next meeting
    \pause
    \item We hope to continue this pattern for the rest of this year
\end{itemize}
\end{frame}

\section{Definitions}

\begin{frame}
\frametitle{Brachistochrone}
\framesubtitle{What Do I mean?}
\begin{itemize}
    \item Etymology
        \pause
        \begin{itemize}
            \item Brachistos ($ \beta \rho \alpha \chi \iota \sigma \tau \sigma $) means "shortest"
            \pause
            \item Chronos ($\chi \rho o \nu o \sigma $) means "time"
        \end{itemize}
    \pause
    \item A brachistochrone curve is the path such that a ball traveling along this path takes the least amount of time
    \pause
    \item This is our problem
    \pause
    \item Formal problem statement
    \begin{itemize}
        \pause
        \item Constraints: given two points $P_1 (x_1, y_1)$ and $P_2 (x_2, y_2)$
        \pause
        \item Find function $y = f(x)$ such that the time it takes for a ball to travel under the influence of gravity from $P_1$ to $P_2$
    \end{itemize}
\end{itemize}
\end{frame}

\section{Getting Started}

\begin{frame}
\frametitle{Getting Started}
\begin{itemize}
    \item Let $s$ be a position vector
    \pause
    \item Let $v$ be the associated velocity vector
    \pause
    \item From last lecture, recall that
    $$
    v = \frac{ds}{dt} \Rightarrow dt = \frac{ds}{v} \Rightarrow t_{12} = \int \limits_{P_1}^{P_2} \frac{ds}{v}
    $$
\end{itemize}
\end{frame}

\begin{frame}
\frametitle{Energy Conservation}
\begin{itemize}
    \pause
    \item Kinetic energy $K = \frac{1}{2} mv^2$
    \pause
    \item Gravitational potential energy $U = mgy$
    \pause
    \item Conservation of energy means that these two are equal
    \pause
    $$
    \frac{1}{2} mv^2 = mgy \Rightarrow v = \sqrt{2gy}
    $$
    \pause
    \item We can substitute this into the last equation
\end{itemize}
\end{frame}

\begin{frame}
\frametitle{Pythagorean Theorem}
\begin{align*}
    ds^2 &= dx^2 + dy^2 \\
    ds^2 &= dx^2 \left ( 1 + \left ( \frac{dy^2}{dx^2} \right) \right) \\
    ds^2 &= dx^2 \left ( 1 + \left ( \frac{dy}{dx} \right)^2 \right) \\
    ds^2 &= dx^2 \left ( 1 + y^{'2} \right ) \\
    ds &= dx \sqrt{1 + y^{'2}} \\
\end{align*}
\end{frame}

\begin{frame}
\frametitle{Putting It All Together}
\begin{itemize}
    \item Original equation:
    $$
    t_{12} = \int \limits_{P_1}^{P_2} \frac{ds}{v}
    $$
    \pause
    \item Conservation of energy:
    $$
    v = \sqrt{2gy}
    $$
    \pause
    \item Pythagorean theorem:
    $$
    ds = dx \sqrt{1 + y^{'2}}
    $$
\end{itemize}
\end{frame}

\section{Lagrangians}
\begin{frame}
\frametitle{Lagrangians}
    $$
    t_{12} = \int \limits_{P_1}^{P_2} \sqrt{\frac{1 + y^{'2}}{2gy}} dx
    $$
\pause
\begin{itemize}
    \item We want to minimize this by...
    \pause
    \item picking a function $y = f(x)$ to minimize integral
    \pause
    \item How do we do it???
    \pause
    \item Lagrangians
\end{itemize}
\end{frame}

\end{document}