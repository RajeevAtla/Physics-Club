\documentclass[11pt]{scrartcl}
\usepackage[utf8]{inputenc}
\usepackage{mathtools}
\usepackage{amssymb}
\usepackage{fancyhdr}
\usepackage{answers}
\usepackage{lastpage}
\usepackage{datetime}
\usepackage{titlesec}
\usepackage{makeidx}
\usepackage{graphicx}
\usepackage{bm}
\usepackage{evan}[fancy, sexy, hdr, colorsec]
\usepackage[dvipsnames]{xcolor}
\usepackage[margin = 1in, letterpaper, portrait]{geometry}
\usepackage[version = 4]{mhchem}

\everymath{\displaystyle}

\pagestyle{fancy}
\rhead{Last Updated: \today}
\lhead{Kinematics Problem Set}
\cfoot{Page \thepage\ of \pageref*{LastPage}}

\renewcommand*\contentsname{\S Table of Contents}

\makeindex


\begin{document}

\titleformat{\section}{\normalfont\Large\bfseries}{\color{red}\S \thesection}{0.5em}{}
\titleformat{\subsection}{\normalfont\Large\bfseries}{\color{olive}\S \thesubsection}{0.5em}{}
\titleformat{\subsubsection}{\normalfont\Large\bfseries}{\color{blue}\S \thesubsubsection}{0.5em}{}

\begin{center}
    \Large \textbf{Kinematics Problem Set}
\end{center}
\begin{center}
    \Large Rajeev Atla
\end{center}

\begin{itemize}
    \item Problem weights
    \begin{itemize}
        \item Problem $n$ is worth $n$ points
        \item So problem 1 is worth 1 point, problem 2 is worth 2 points, etc.
        \item Total of 15 points
    \end{itemize}
    \item Try to get as many points as you can!
    \item Dont forget to have fun!
    \item Feel free to reach out for hints
    \item There will be a leaderboard
\end{itemize}

\newpage

\section{Problem 1}
Using calculus and/or geometry, derive the equation

$$
x(t) = x_0 + v_0 t + \frac{1}{2} at^2
$$

\subsection*{Solution}
We go the calculus route.
We know that $v(t) = v_0 + at$ and that $x = \int v(t)\ dt$.
Carrying out the integral using the power rule, we see that

$$
x(t) = C + v_0 t + \frac{1}{2} at^2
$$

Substituting $t = 0$, we see that the constant $C$ must be equal to the intitial value of $x$, recovering the equation.

\newpage

\section{Problem 2}
For two vectors $\bm{a}$ and $\bm{b}$, prove the following inequalities:

$$
|\bm{a}| - |\bm{b}| \leq |\bm{a}+\bm{b}| \leq |\bm{a}|+ |\bm{b}|
$$

\subsection*{Solution}
This problem can actually be done without much algebra.
We consider the wording, which is eerily similar to the triangle inequality.
In fact, its actually the triangle inequality for vectors.

First, lets look at some edge cases.
Suppose both $\bm{a}$ and $\bm{b}$ are vectors in the same direction (parallel), making sort of a degenerate triangle.
Drawing it out, we can see that the two vectors add one dimensionally with a total magnitude of $|\bm{a}|+ |\bm{b}|$.

Another edge case we can consider will be when the two vectors are in opposing directions (known as antiparallel), making yet anothe degenerate triangle.
This means that magnitude of the final vector will be $|\bm{a}| - |\bm{b}|$.

These two edge cases form the equality cases of our inequality.
To complete the proof, we use the cosine rule and see that the inequality holds even in non-edge cases.
Suppose we know the angle between the two vectors to be $\theta$.
The cosine rule states that the magnitude of the sum will obey

$$
|\bm{a}+\bm{b}| = |\bm{a}|^2 + |\bm{b}|^2 - 2|\bm{a}||\bm{b}| \cos{\theta}
$$

Substituting $\theta = \pi$ and $\theta = \frac{\pi}{2}$, we both recover our edge cases and see that the inequality holds true. 

\newpage

\section{Problem 3}
A ball is thrown from the ground.
The ball crosses the height $h_1$ twice, with $T_1$ seconds between crossings.
Above, at a height of $h_2$, the ball takes $T_2$ seconds between crossings.
Derive an expression for $g$, the acceleration due to gravity, in terms of these variables.

\subsection*{Solution}
This is a pretty algebra-heavy problem.
Let the initial velocity of the ball be $v_0$, pointed upwards.
Using kinematics, we know that

$$
y(t) = v_0 t - \frac{1}{2} gt^2
$$

At a height $h_1$, the times at which the ball is at this height will be (using the quadratic formula)

$$
t_{\pm 1} = \frac{v_0 \pm \sqrt{v_0^2 - 2gh_1}}{g}
$$

Similarly, we can also solve for the times at which the ball will be at $t_2$, getting

$$
t_{\pm 2} = \frac{v_0 \pm \sqrt{v_0^2 - 2gh_2}}{g}
$$

We can now find $T_1$ and $T_2$ by taking $t_{+1} - t_{-1}$ and $t_{+2} - t_{-2}$, respectively.
Doing so, we find

$$
T_1 = \frac{2 \sqrt{v_0^2 - 2gh_1}}{g}
$$

$$
T_2 = \frac{2 \sqrt{v_0^2 - 2gh_2}}{g}
$$

The one variable we introduced is $v_0$, so to eliminate, we solve for $v_0$ in both equations.
Doing so gives us

$$
v_0^2 = \frac{1}{4} gT_1^2 + 2gh_1 = \frac{1}{4} gT_2^2 + 2gh_2
$$

We now have given ourselves an equation we can solve for $g$.
Doing so we get,

$$g = \boxed{\frac{8(h_2 - h_1)}{T_1^2 - T_2^2}}$$

\newpage

\section{Problem 4}
Bobby wants to swim across a river of width $w$.
This river flows east to west with a velocity of $v_r$.
In still water, Bobby can move in any direction with a speed of $v_b$.
In what direction should Bobby move to minimize the total distance he travels.
Hint: There are two cases, check both of them.

\newpage

\section{Problem 5}
A rabbit is at the origin and a fox is at $(0, -a)$.
At $t=0$, the rabbit begins moving with a velocity $\bm{v} = v \hat{x}$.
Simultaneously, the fox begins running directly in the direction of the rabbit with speed $v$.
After a long time, the distance between the two animals is $d$.
Find $d$.


\end{document}